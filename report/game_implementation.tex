% \let\clearpage\relax
% \bigskip
% \bigskip
\section{Game Implementation}

All components of the game were implemented in Python. While Python is not the fastest language due to its dynamically typed nature, it allows for rapid prototyping and short development cycles. Furthermore, compilation using e.g. \textit{Cython} can improve run-time to get results closer to those of statically typed languages.

\subsection{Software Architecture}

The game was implemented following a Model-View-Controller achitecture. Additionally, the AI player has been implemented as a separate class of its own where each instance is handed a game instance. This architecture was chosen to achieve greater modularity of the components which allowed for more flexibility when trying out new approaches in the development process. As a result, the classes making up the game are:

\begin{itemize}
    \item \textit{Game\_Logic}: Game state, legal moves and win conditions
    \item \textit{View}: User interface
    \item \textit{Controller}: Controller Game initialization and play out
    \item \textit{AI}: Search based player instance
\end{itemize}

\begin{figure}[h]
    \centering
    \includegraphics[scale=0.5]{MVC.png}
    \caption{Classes (conceptualized)}
\end{figure}

The modules \textit{Game\_Logic} and \textit{View} are implemented in an object-oriented fashion. Despite the possibility to access all class attributes directly, several getter and setter methods are used for a more consistent access pattern for similar queries of the game state. The \textit{Controller} is implemented as a script following the requirements of \textit{Pygame}'s game loop. 


\subsection{Data Structures}
The game itself is represented as a Python dictionary containing the grid, turn and winner of a game. The grid in turn is represented as a nested list of tiles. tiles are again represented as dictionaries used to save the row, column and owner of each tile. In an earlier prototype, the game and tile were implemeted as classes, however this proved to be unnecessary overhead. The use of dictionaries allows for easy key-based access, iteration over elements and a simple way of manipulating the game state.

\subsection{Game Actions}

\subsubsection{Legal Moves}
The game rules are implelemented as part of the Controller module. Implementationally, the action of placing a stone is separated from its pre-and post-conditions with each being encoded in a function of their own. Any such function takes a game and information describing the action as a parameter. 
% The basic outline is as follows:
% \begin{verbatim}
%     function game_action(game, action) is
%         if pre_condition(game, action) then
%             return perform_action(game, action)
%         else return None
% \end{verbatim}
Whether a move is legal or not is checked based on the turn and neighbours of the tile about to be placed. The neighbours are identified according to their index in the nested lists making up the game grid.


\subsubsection{Win Conditions}

\textbf{Five in a Row}\\
The criterion for five in a row is implemented recursively. Each tile will recursively query neighbours in a straight line to count how many continuous tiles of the same color there are.

\textbf{Surround}\\
The surround win criterion is implemented in a depth-first floodfill fashion. Each tile recursively queries its neighbours to check whether there is a way to get to the border of the game board. If there is no such path without encountering a tile of enemy color, a surround is detected.


\subsection{External Libraries}
To save development time, a number of external public python libraries were used for the project. Most prominently, the graphical user interface is built entirely on top of the \textit{Pygame} engine. Python's \textit{logging} library was used extensively for debugging the game AI. To avoid undesired side-effects when copying dictionaries, \textit{copy}'s \textit{deepcopy} is used for copying nested mutable objects.Other external Libraries used in the project include \textit{math} and \textit{time}.