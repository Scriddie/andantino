\section{Search Techniques}

\subsection{Alpha-Beta-Search}
%  as introduced by \cite{knuth1975analysis}
The game AI used for this game is implemented using an alpha-beta format, more specifically, the Negamax algorithm. Alpha-Beta pruning is used to reduce the number of nodes explored while maintaining the quality of the evaluation. The AI's next move of choice is reset to the move in question, whenever there is an update of the evaluation at the very top of the search tree.

\subsection{Debugging}
To check the funcionality of the search techniques mentioned above, a debugging log as can be seen in \ref{fig:debugging} was used to keep track of the game tree. This proved to be a useful tool for ensuring correct evaluation values and prunings.
\begin{figure}[h]
    \centering
    \includegraphics[scale=0.6]{log3.png}
    \caption{Debugging log excerpt}
    \label{fig:debugging}
\end{figure}  

\subsection{Evaluation Function}
States with a definitive win are assigned a value greater than any other that can possibly be reached, plus their depth in their search tree to find the shortest win. Losing states are evaluated in the same way with an opposite sign. The evaluation of any other position relies on rewarding combinations of continuous lines. Each stone is assigned a value amounting to the sum of the lenghts of all continuous lines it is a part of in any of the three directions of the hexagonal board. In this method of evaluation, adding a stone to a continuous line of other connected stones will increase the evaluation not only by the value of the new stone, but also by the increase in value of the pre-existing ones. This results in an increasing marginal utility of continuous lines. The averages of these values per number of stones placed are taken for either player and subracted to get a comparative evaluation:
\begin{center}
    \begin{math}
        \frac{\sum_{own\_tiles}\sum_{directions}c}{\sum{own\_tiles}} 
        - \frac{\sum_{opponent\_tiles}\sum_{directions}c}{\sum{opponent\_tiles}}
    \end{math}
\end{center}
Where $c$ denotes the lenth of the continuous line in a certain direction which the tile is a part of.

For example, in \ref{fig:tree_in_row}, the black player will evaluate the position as $\frac{3 + 3 + 3}{3} - \frac{2 + 2}{2} = 0$. Intuitively speaking, the black player does not consider this configuration advantageous since both players have been following the same strategy of laying straight lines.
\begin{figure}[H]
    \centering
    \includegraphics[scale=0.9]{three_in_row.png}
    \caption{Evaluation of continuous lines}
    \label{fig:tree_in_row}
\end{figure}

% In \ref{fig:evaluation}, the white player has chosen to not go for continuous lines but rather cover more continuous area without giving black a chance to get to four in a row. The black player's setup however, does not maximize its continuous are. Its zero ply evaluation of the position is therefore $\frac{(3 + 3 + 3 + 2 + 2)}{4} - \frac{2 + 2 + 2 + 2 + 2 + 2}{3} = -0.75$.
% \begin{figure}[H]
%     \centering
%     \includegraphics{evaluation.png}
%     \caption{Strategy divergence}
%     \label{fig:evaluation}
% \end{figure}


\subsection{Performance}
 The evaluation function described in this paragraph favours placing long continuous lines whenever possible, with a tendency to maximize connected area if there is no obvious way of achieving lines of greater length. Since a large continuous area will usually result in mutliple possibilities to achieve n-in-a-row, this approach seems to serve the game objective well. Whenever a player using this evaluation function has the initiative, it will often go for simple wins, forcing the opponent to block. This results in disconnected chains and isolated stones for the opponent which allow for pressure play by threatening a surround win. When the AI player does not have the initiative, it will often try to achieve a large connected area with a high number of chains that are hard to block simultaneously, until it can gain the initiative. This method of evaluation was choosen after it proved able to beat a number of other hand-crafted evaluation functions.